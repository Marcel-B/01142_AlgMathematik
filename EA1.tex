\documentclass[12pt]{article}
\usepackage[utf8]{inputenc}
\usepackage[T1]{fontenc}
\usepackage{lmodern}
\usepackage{ngerman}
\usepackage{amsmath}
\usepackage{amssymb}
\usepackage{array}
\usepackage{german,fancyheadings}
\pagestyle{fancy}
%\setlength{\parindent}{0em} 
\newcommand{\N}{\mathbb{N}}
\newcommand{\M}{$\times$}
\newtheorem{Sa}{Satz}[subsection]
\newtheorem{Kor}{Korollar}[subsection]
\newtheorem{Prop}{Proposition}[subsection]
\newtheorem{Def}{Definition}[subsection]
\lhead{Marcel Benders - Matrikelnummer: 5431760}


\begin{document}

\section{E 1.1}
--

\section{E 1.2}
\subsection{a}
Die Zerlegung in disjunkte Zyklen sieht folgendermaßen aus:

\begin{equation}
\sigma_1 =
\bigl(\begin{smallmatrix}
1 & 3 & 9 & 5  \\
3 & 9 & 5 & 16 
\end{smallmatrix}\bigr) = \bigl  < 1 \; 3\; 9\; 5\; 16 \bigr >
\end{equation}

\begin{equation}
\sigma_2 =
\bigl(\begin{smallmatrix}
2  & 13 \\
13 & 11
\end{smallmatrix}\bigr) = \bigl < 2\; 13\; 11 \bigr >
\end{equation}

\begin{equation}
\sigma_3 =
\bigl(\begin{smallmatrix}
4  \\
18
\end{smallmatrix}\bigr) = \bigl < 4\; 18 \bigr >
\end{equation}

\begin{equation}
\sigma_4 =
\bigl(\begin{smallmatrix}
6  & 20 & 14 & 19 & 21\\
20 & 14 & 19 & 21 & 12
\end{smallmatrix}\bigr) = \bigl < 6 \; 20\; 14\; 19\; 21\; 12 \bigr >
\end{equation}

\begin{equation}
\sigma_5 =
\bigl(\begin{smallmatrix}
7  & 10 \\
10 & 15
\end{smallmatrix}\bigr) = \bigl < 7\; 10\; 15 \bigr >
\end{equation}

\begin{equation}
\sigma_6 =
\bigl(\begin{smallmatrix}
8 \\
8
\end{smallmatrix}\bigr) = \bigl < 8 \bigr >
\end{equation}

\begin{equation}
\sigma_7 =
\bigl(\begin{smallmatrix}
17 \\
17
\end{smallmatrix}\bigr) = \big < 17 \big >
\end{equation}

\begin{equation}
\sigma_8 =
\bigl(\begin{smallmatrix}
22 \\
22
\end{smallmatrix}\bigr) = \bigl < 22 \bigr >
\end{equation}

\section{E 1.3}
\subsection{a}
--
\subsection{b}
Frage: Wieviele Hochzeiten (bezüglich $F$ und $M$) gibt es (abhängig von $n$)? \\

Die Kardinalität beider Mengen $F$ und $M$ ist $n$, also $|F| =|M| = n$. Aus diesen beiden Mengen
lassen sich $n$ disjunkte Paarmengen bilden, also gibt es $n$ Hochzeiten.


Möchte man die Möglichkeiten verschiedener Hochzeiten - wenn alle Teilnehmer heiraten -
 wissen, berechnet man die Fakultät, also $n!$ .

\subsection{c}
Es ist zu Zeigen wie viele Möglichkeiten es gibt. Bei dem Spiel werden $n$ Fragen gestellt. Angenommen
man kennt die richtigen Antworten und darf nicht falsch antworten.Dann kann entweder mit ja,nein oder
aussetzen geantwortet werden also $k=3$. Somit folgt, dass es $k^n$ verschiedene Möglichkeiten gibt dieses
Spiel zu spielen. 

\section{E 1.4}
--

\section{E 1.5}
--


\section{E 1.6}
\subsection{a}
Es haben 150 Personen an der Klausur teilgenommen. Daraus ergibt sich folgendes:
\begin{equation}
\frac{150}{365} \approx 0,4109 \approx 41 \%.
\end{equation}
Die Wahrscheinlickeit beträgt $\approx 41\%$.

\subsection{b}
Zuerst wird ermittelt, wie groß die Wahrscheinlickeit an einem Korrekturtag ist. An einem Tag werden
30 Klausuren korrigiert. Daraus ergibt sich folgendes:

\begin{equation}
\frac{30}{365} \approx 0,08219. 
\end{equation}

Es besteht somit die Wahrscheinlickeit von $\approx 8\%$, dass jemand von den 30 Teilnehmern am
selben Tag Geburtstag hat. Dies muss jetzt noch in Relation zu den 5 Tagen gesetzt werden:

\begin{equation}
\frac{1}{5} = 0,2.
\end{equation}


Nun werden die beiden Werte Multipliziert:

\begin{equation}
0,2 \cdot 0,08219 \approx 0,0164.
\end{equation}

Somit hat man eine Wahrscheinlickeit von $\approx 1,6\%$, dass an jedem der 5 Korrekturtage mindestens
einer der Teilnehmer am selben Tag wie der Korrektor Geburtstag hat.

\section{E 1.7}
Karten $n=5, \text{asse}=2, \text{luschen}=3$ zwei Karten ziehen und nicht zurücklegen. Es muss der Binomialkoeffizient
benutzt werden.

Es gibt 

\begin{equation}
\binom{5}{2} = \frac{5!}{2!(5-2)!} = \frac{ 4 \cdot 5}{2} = 10
\end{equation}

verschiedene Möglichkeiten zwei Karten aus dem 5er Set zu ziehen.

\subsection{a}
\begin{itemize}
    \item Die Wahrscheinlickeit für zwei Asse ist
    \begin{equation}
    \frac{2}{5} = 0,4 = 40\%.
    \end{equation}

\end{itemize}


\end{document}