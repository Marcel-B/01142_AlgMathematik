\documentclass[12pt]{article}
\usepackage[utf8]{inputenc}
\usepackage[T1]{fontenc}
\usepackage{lmodern}
\usepackage{ngerman}
\usepackage{amsmath}
\usepackage{amssymb}
\usepackage{array}
\usepackage{german,fancyheadings}
\usepackage{eurosym}
\pagestyle{fancy}
\setlength{\parindent}{0em} 
\newcommand{\N}{\mathbb{N}}
\newcommand{\M}{$\times$}
\newtheorem{Sa}{Satz}[subsection]
\newtheorem{Kor}{Korollar}[subsection]
\newtheorem{Prop}{Proposition}[subsection]
\newtheorem{Def}{Definition}[subsection]
\lhead{Marcel Benders - Matrikelnummer: 5431760}


\begin{document}

\section*{E 1.1}
--

\section*{E 1.2}
\subsection*{a}
Die Zerlegung in disjunkte Zyklen sieht folgendermaßen aus:

\begin{equation*}
\sigma_1 = \bigl( \begin{smallmatrix}
1 & 3 & 9 & 5  \\
3 & 9 & 5 & 16 
\end{smallmatrix}\bigr) = \bigl  < 1 \; 3\; 9\; 5\; 16 \bigr > 
\end{equation*}

\begin{equation*}
\sigma_2 = \bigl( \begin{smallmatrix}
2  & 13 \\
13 & 11
\end{smallmatrix} \bigr) = \bigl < 2\; 13\; 11 \bigr >
\end{equation*}

\begin{equation*}
\sigma_3 =
\bigl(\begin{smallmatrix}
4  \\
18
\end{smallmatrix}\bigr) = \bigl < 4\; 18 \bigr >
\end{equation*}

\begin{equation*}
\sigma_4 =
\bigl(\begin{smallmatrix}
6  & 20 & 14 & 19 & 21\\
20 & 14 & 19 & 21 & 12
\end{smallmatrix}\bigr) = \bigl < 6 \; 20\; 14\; 19\; 21\; 12 \bigr >
\end{equation*}

\begin{equation*}
\sigma_5 =
\bigl(\begin{smallmatrix}
7  & 10 \\
10 & 15
\end{smallmatrix}\bigr) = \bigl < 7\; 10\; 15 \bigr >
\end{equation*}

\begin{equation*}
\sigma_6 =
\bigl(\begin{smallmatrix}
8 \\
8
\end{smallmatrix}\bigr) = \bigl < 8 \bigr >
\end{equation*}

\begin{equation*}
\sigma_7 =
\bigl(\begin{smallmatrix}
17 \\
17
\end{smallmatrix}\bigr) = \big < 17 \big >
\end{equation*}

\begin{equation*}
\sigma_8 =
\bigl(\begin{smallmatrix}
22 \\
22
\end{smallmatrix}\bigr) = \bigl < 22 \bigr >
\end{equation*}

\subsection*{b}
Zu Zeigen ist die Zerlegung von $\binom{26}{11}$ in Primfaktoren. Zuerst setzen wir die Zahlen
ein und erhalten somit
\begin{equation*}
\binom{26}{11} = \frac{26!}{11!(26-11)!} = \frac{26!}{11!15!}.
\end{equation*}
Anschließend kürzen wir die Fakultäten und erhalten ausgeschrieben folgenden Bruch:
\begin{equation*}
\frac{16 \cdot 17 \cdot 18 \cdot 19 \cdot 20 \cdot 21 \cdot 22 \cdot 23 \cdot 24 \cdot 25 
\cdot 26}{2 \cdot 3 \cdot 4 \cdot 5 \cdot 6 \cdot 7 \cdot 8 \cdot 9 \cdot 10 \cdot 11}.
\end{equation*}

Durch geschicktes kürzen erhalten wir folgende Faktoren:
\begin{equation*}
2 \cdot 17 \cdot 3 \cdot 19 \cdot 2 \cdot 23 \cdot 3 \cdot 5 \cdot 26.
\end{equation*}

Nun muss noch die 26 zerlegt werden und wir erhalten somit mit
\begin{equation*}
2 \cdot 2 \cdot 2 \cdot 3 \cdot 3 \cdot 5 \cdot 13 \cdot 17 \cdot 19 \cdot 23 = 2^3  \cdot  3^2  \cdot 5  \cdot 13 \cdot 17 \cdot 19 \cdot 23.
\end{equation*}
die gewünschte Primfaktorzerlegung.
\subsection*{c}
--

\subsection*{d}
--

\section*{E 1.3}
\subsection*{a}
--

\subsection*{b}
Frage: Wieviele Hochzeiten (bezüglich $F$ und $M$) gibt es (abhängig von $n$)? \\

Die Kardinalität beider Mengen $F$ und $M$ ist $n$, also $|F| =|M| = n$. Aus diesen beiden Mengen
lassen sich $n$ disjunkte Paarmengen bilden, also gibt es $n$ Hochzeiten.


Möchte man die Möglichkeiten verschiedener Hochzeiten - wenn alle Teilnehmer heiraten -
 wissen, berechnet man die Fakultät, also $n!$ .

\subsection*{c}
Es ist zu Zeigen wie viele Möglichkeiten es gibt. Bei dem Spiel werden $n$ Fragen gestellt. Angenommen
man kennt die richtigen Antworten und darf nicht falsch antworten.Dann kann entweder mit ja,nein oder
aussetzen geantwortet werden also $k=3$. Somit folgt, dass es $k^n$ verschiedene Möglichkeiten gibt dieses
Spiel zu spielen. 

\section*{E 1.4}
--

\section*{E 1.5}
--


\section*{E 1.6}
\subsection*{a}
Es haben 150 Personen an der Klausur teilgenommen. Daraus ergibt sich folgendes:
\begin{equation*}
\frac{150}{365} \approx 0,4109 \approx 41 \%.
\end{equation*}
Die Wahrscheinlickeit beträgt $\approx 41\%$.

\subsection*{b}
Zuerst wird ermittelt, wie groß die Wahrscheinlickeit an einem Korrekturtag ist. An einem Tag werden
30 Klausuren korrigiert. Daraus ergibt sich folgendes:

\begin{equation*}
\frac{30}{365} \approx 0,08219. 
\end{equation*}

Es besteht somit die Wahrscheinlickeit von $\approx 8\%$, dass jemand von den 30 Teilnehmern am
selben Tag Geburtstag hat. Dies muss jetzt noch in Relation zu den 5 Tagen gesetzt werden:

\begin{equation*}
\frac{1}{5} = 0,2.
\end{equation*}


Nun werden die beiden Werte Multipliziert:

\begin{equation*}
\frac{30}{365} \cdot \frac{1}{5} = \frac{30}{365 \cdot 5} = \frac{30}{1825} \approx 0,01644.
\end{equation*}

Somit hat man eine Wahrscheinlickeit von $\approx 1,6\%$, dass an jedem der 5 Korrekturtage mindestens
einer der Teilnehmer am selben Tag wie der Korrektor Geburtstag hat.

\section*{E 1.7}
Die Kardinalität der Menge aus der wir 2 Karten ziehen beträgt 5. Wir ziehen zeitgleich. Es muss der Binomialkoeffizient
benutzt werden.

Daraus ergeben sich

\begin{equation*}
|\Omega| = \binom{5}{2} = \frac{5!}{2!(5-2)!} = \frac{ 4 \cdot 5}{2} = 10
\end{equation*}

verschiedene Möglichkeiten zwei Karten aus dem 5er Set zu ziehen.

\subsection*{a}
\begin{itemize}
    \item Nun muss die Wahrscheinlichkeit des Ereignisses "`es werden zwei Asse gezogen"' berechnet 
    werden. Die Möglichkeiten aus 2 Assen 2 Asse zu ziehen beträgt
    
    \begin{equation*}
    |A| = \binom{2}{2} = 1.
    \end{equation*}

    Insgesamt liegt die Wahrscheinlichkeit bei
    \begin{equation*}
    P(A) = \frac{|A|}{| \Omega |} = \frac{1}{10} = 0,1
    \end{equation*}
    also 10\%, 2 Asse aus dem gemischten Kartenset zu ziehen.

    \item Es folgt die Berechnung der Wahrscheinlichkeit "`es werden zwei Luschen gezogen"'.
    Die Möglichkeiten aus 3 Luschen 2 zu ziehen beträgt

    \begin{equation*}
    |A| = \binom{3}{2} = 3.
    \end{equation*}

    Die Wahrscheinlichkeit liegt somit bei
    \begin{equation*}
    P(A) = \frac{|A|}{| \Omega |} = \frac{3}{10} = 0,3,
    \end{equation*}
    also 30\%, 3 Luschen aus dem gemischten Kartenset zu ziehen.

    \item Und nun die Berechnung der Wahrscheinlichkeit "`genau ein Ass und eine Lusche werden gezogen"'.
    Dazu subtrahieren wir die beiden bisherigen Ergebnisse von 1:
    \begin{equation*}
    1 - 0,3 - 0,1 = 0,6.
    \end{equation*}
    Wir erhalten also eine 60\%ige Wahrscheinlichkeit, 1 Lusche und 1 Ass aus dem gemischten Kartenset zu ziehen.

\end{itemize}

\subsection*{b}
Dazu muss der Erwartungswert berechnet werden. Es gibt 3 mögliche Gewinne/Verluste mit den
Werten 20, 5, -30. Diese treten mit den Wahrscheinlichkeiten 0,1, 0,6 und 0,3 auf. Der 
Erwartungswert $\mu$ ist somit

\begin{equation*}
\mu = 0,1 \cdot 20 + 0,6 \cdot 5 + 0,3 \cdot -30 = -4.
\end{equation*}

Spielen wir sehr oft, verlieren wir \EUR{4} pro Spiel.
\subsection*{c}
In der zweiten Runde ändert sich die Wahrscheinlichkeit. Im Spiel verbleiben 1 Ass und 3 Luschen.
Die Wahrscheinlichkeit das Ass zu ziehen beträgt

\begin{equation*}
\frac{1}{4} = 0,25.
\end{equation*}

Dementsprechend liegt die Wahrscheinlichkeit eine Lusche zu ziehen bei 0,75. Der Erwartungswert $\mu$ setzt sich nun wiefolgt zusammen:

\begin{equation*}
\mu = \underbrace{ 0,1 \cdot 20}_{\text{zwei Asse}} + \underbrace{(0,6 \cdot 0,75) \cdot 5}_{\text{zwei Luschen 2. Versuch}} +  \underbrace{(0,6 \cdot 0,25) \cdot (20+5)}_{\text{zwei Asse 2. Versuch}}   + \underbrace{0,3 \cdot -30}_{\text{zwei Luschen}}  = -1.
\end{equation*}

Bei dieser Regel Beträgt der Verlust pro Spiel bei vielen Spielen \EUR{1}.




\subsection*{d}
Fogendes haben wir in Teil b errechnet:

\begin{eqnarray*}
\mu &=& 0,1 \cdot 20 + 0,6 \cdot 5 + 0,3 \cdot -30  \\
    &=& 2 + 3 + -9  \\
    &=& 5 - 9 \\
    &=& -4.
\end{eqnarray*}


Damit das Spiel fair ist, brauchen wir anstatt der -9 eine -5. 
\begin{eqnarray*}
0,3 \cdot x &=& -5 \\
x           &=& \frac{-5}{0,3} \\
x           &=& -16 \frac{2}{3}.
\end{eqnarray*}

Das Ergebnis $-16 \frac{2}{3}$ ist rational -- jedoch nicht realistisch.
\end{document}